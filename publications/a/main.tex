\chapter{Set Title of Your First Paper}
% Settings for this paper
\renewcommand{\folder}{publications/a}
\graphicspath{{\folder/gfx/}}


\begin{raggedright}
	\renewcommand*{\thefootnote}{*}
	Remus Mihail Prunescu$^1$, Mogens Blanke$^1$, Jon Geest Jakobsen$^2$, Gürkan Sin\footnote{Principal corresponding author. Tel.: +45 45252806; E-mail: gsi@kt.dtu.dk}$^3$\\[2em]
	$^1$Department of Electrical Engineering, Automation and Control Group, Technical University of Denmark, Elektrovej Building 326, 2800, Kgs. Lyngby, Denmark\\[.5em]
	$^2$Department of Process Control and Optimization, DONG Energy Thermal Power A/S, Nesa Allé 1, 2820, Gentofte, Denmark\\[.5em]
	$^3$CAPEC-PROCESS, Department of Chemical and Biochemical Engineering, Technical University of Denmark, Søltofts Plads Buildings 227 and 229, 2800, Kgs. Lyngby, Denmark
	\renewcommand*{\thefootnote}{\arabic{footnote}}
\end{raggedright}

\vfill
{\noindent\bfseries Abstract:}

\noindent Second generation biorefineries transform lignocellulosic biomass into chemicals with higher added value following a conversion mechanism split into: pretreatment, enzymatic hydrolysis, fermentation, and purification. The objective of this study is to identify the optimal operational point of the refinery such that economic profit is maximized. The decision variables are identified as: pretreatment temperature, enzyme dosage in enzymatic hydrolysis, and yeast charge per batch in fermentation. The plant is treated in an integrated manner taking into account the interactions and trade-offs between the conversion steps. A sensitivity and uncertainty analysis follows for the optimal solution considering both model and feed parameters. It is found that an optimization supervisory layer is superior to a traditional refinery operation, and also reduces the uncertainty on the profit curve.

\vfill

\newpage
\section{Introduction}
Second generation lignocellulosic biorefineries reached commercial reality in 2012 \citep{Larsen2012}, and consequently many large scale plants are in operation nowadays: e.g. Beta Renewables, Abengoa Bioenergy, GranBio, POET-DSM, etc \citep{Brown2015}. Most biorefineries produce bioethanol but the drop in oil price reduced the demand on the biofuel. However, plant upgrades for chemicals with higher-added values are recommended \citep{Cheali2015} making biorefineries still competitive in an oil dependent environment.